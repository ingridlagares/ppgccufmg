Informação geoespacial é um dos elementos mais críticos para tomadas de decisão em várias disciplinas. Alguns exemplos de áreas que se beneficiam da informação geoespacial são transporte, uso do solo, hidrologia, urbanismo, saúde e meio ambiente. Essa interdisciplinaridade entre várias organizações depende da disponibilidade e acessibilidade de dados espaciais consistentes e de qualidade \citep{abbas}.

Compartilhamento, troca e integração de dados tornaram-se, há muito tempo, questões importantes em Sistemas de Informação Geográficos (SIG). De acordo com \cite{goodchild2012interoperating}, a interoperabilidade traz diversos benefícios, como redução de custos, aumento de flexibilidade e projetos mais rápidos. Isso significa maior abertura na indústria de software e disseminação de sistemas com interfaces intuitivas e amigáveis a interoperabilidade permite que desenvolvedores construam aplicativos que integrem com esses componentes já existentes. 

A ampla disponibilidade de dados é também de importância fundamental para projetos que envolvem informação geográfica. Na grande maioria desses projetos, alguma informação geográfica básica é necessária, além de dados específicos do projeto. Por exemplo, projetos urbanos se beneficiam da disponibilidade de dados urbanos básicos, como arruamento, áreas de proteção e elementos de infraestrutura urbana. Desde o início da década de 2010, dados geográficos são compartilhados por meio de infraestruturas de dados espaciais (IDEs) \citep{masser2019geographic}. IDEs, como a Infraestrutura Nacional de Dados Espaciais (INDE) e outras em todo o mundo, permitem o acesso direto a dados geográficos usando serviços Web padronizados pela ISO, a partir de iniciativas do Open Geospatial Consortium (OGC). IDEs são tipicamente associadas a entidades cartográficas nacionais, como o IBGE no Brasil, ou a órgãos produtores de dados governamentais abertos. Algumas IDEs, por outro lado, surgiram de iniciativas localizadas e com temática específica.

Uma dessas IDEs foi criada para o Projeto Brumadinho UFMG\footnote{\url{http://projetobrumadinho.ufmg.br/}}
 em decorrência da ação judicial deflagrada após o rompimento da barragem da Mina do Córrego do Feijão, em Brumadinho (MG), em 2019. O Juízo responsável pelos processos legais de reparação, diante da necessidade de avaliar os impactos sociais, ambientais e necessidades da população afetada, firmou acordo com a UFMG no sentido de produzir dados sobre a região do desastre, e organizar esses dados em uma plataforma de dados, denominada Plataforma Brumadinho UFMG. A Plataforma Brumadinho UFMG combina o gerenciamento de dados não estruturados (documentos de texto) e dados estruturados (geoespaciais e científicos). O objetivo é prover acesso amplo à população, às partes do processo e demais interessados, bem como a pesquisadores, buscando expor a documentação incluída nos processos legais e os dados de apoio levantados.
 
 Para viabilizar esse amplo acesso, a plataforma segue integralmente os princípios FAIR (\textit{Findable, Accessible, Interoperable, Reusable})  \citep{wilkinson2016fair}, em geral utilizado em relação a dados científicos, porém estendidos pelo projeto para todo o conteúdo organizado. Segundo esses princípios, os dados que fazem parte do acervo da plataforma precisam ser (1) localizáveis, sem viés que privilegie uma fonte sobre outras, (2) acessíveis por qualquer cidadão, (3) interoperáveis, em formato tecnologicamente neutro, (4) reutilizáveis, pois acompanhados de metadados registrados e indexados.
 
 Para garantir os princípios FAIR é necessário que existam padrões, APIs e protocolos para servir os dados. A IDE da Plataforma Brumadinho UFMG utiliza a padronização mais disseminada e utilizada em SIG: a OGC Web Services (OWS), introduzida em 2001 \citep{harrison2001introduction}. Embora tenham tido esforços para atualizar a arquitetura da OWS, ela não acompanhou a evolução dos Web Services modernos \citep{tu2006web}, \citep{borba} observa que as principais IDEs do mundo não estão integradas, destacando a necessidade de avançar para novas tecnologias que enderecem esses serviços de forma adequada à Web. 
 
 Em 2019 a OGC deu início à especificação da OGC API, uma nova família de padrões para disponibilização de dados, dessa vez voltada para APIs REST. Muitos documentos dos módulos da OGC API ainda estão no estágio de esboço, sendo revisados e comentados pela comunidade. Portanto, o atual cenário motiva o presente trabalho, que pretende, a partir de implementações de um Web Service e de uma API que servem dados de uma como a IDE da Plataforma Brumadinho UFMG, registrar as vantagens e desvantagens das duas abordagens para serviço na Web e a relevância de definir uma nova família de padrões para APIs de Sistemas de Informação Geográfica.


% Sequência: problema: interoperabilidade de software e acesso a dados > IDE como recurso de publicação de acervos de dados > projeto Brumadinho e necessidade de disseminação de conteúdo > plataforma Brumadinho > IDE da plataforma Brumadinho > ... > objetivo: APIs em IDEs. Estudo de caso: APIs na IDE da Plataforma Brumadinho.

\section{Objetivo}

O objetivo desta dissertação é realizar uma comparação qualitativa e quantitativa entre a OGC API e os OGC Web Services para Infraestruturas de Dados Espaciais, utilizando como estudo de caso a IDE criada para o Projeto Brumadinho UFMG. Esse trabalho visa analisar as características, funcionalidades e desempenho dessas abordagens com diferentes configurações, visando apoiar a seleção adequada de serviços de interoperabilidade espacial e a discussão sobre o futuro do uso de Web Services e APIs no ambiente de geoinformação.

Uma abordagem metodológica combinada, que inclui tanto a comparação qualitativa baseada em percepções de desenvolvedores e revisão bibliográfica quanto a comparação quantitativa utilizando benchmarking, permitirá uma análise abrangente e fundamentada dos impactos que a OGC API tem potencial de atingir.

Nesse sentido, são definidos os seguintes objetivos específicos:

\begin{itemize}
    \item Realizar uma revisão bibliográfica abrangente sobre os conceitos, princípios e características da OGC API e dos OGC Web Services utilizados em Infraestruturas de Dados Espaciais.
    \item Conduzir um levantamento da percepção de desenvolvedores e profissionais especializados sobre a OGC API e os OGC Web Services, visando identificar o perfil desses profissionais, percepção da OGC API a partir de métricas de software e percepção sobre o futuro da OGC API.
    \item Realizar uma busca sistemática por artigos científicos, relatórios técnicos e estudos de caso que abordem a implementação e utilização da OGC API e dos OGC Web Services em Infraestruturas de Dados Espaciais.
    \item Executar testes de benchmarking para avaliar a performance de API e Web Services em conformidade com os padrões OGC, analisando parâmetros como abordagem do serviço, ambiente de execução, tipo de armazenamento dos dados, tipo e tamanho do dado.
    \item Realizar análises estatísticas, como regressão linear e projeto fatorial (com um fator e com dois fatores), para identificar correlações entre variáveis que afetam a performance e comparar quantitativamente a eficiência e capacidade de resposta da OGC API e dos OGC Web Services.
\end{itemize}

\section{Estrutura do trabalho}

O restante desta dissertação está organizado como se segue. O Capítulo 2 apresenta a fundamentação teórica, fornecendo uma revisão dos conceitos e características da OGC API e dos OGC Web Services. O Capítulo 3 descreve a metodologia adotada, descrevendo as estratégias utilizadas para a comparação qualitativa e quantitativa, incluindo a elaboração do questionário para coleta da percepção dos desenvolvedores sobre a OGC API, a revisão bibliográfica e a realização de testes de benchmarking. O Capítulo 4 apresenta os resultados obtidos, incluindo análises dos dados coletados, além da discussão dos principais achados. O Capítulo 5 aborda brevemente o cronograma para execução do trabalho e o que foi feito. Por fim, o Capítulo 6 conclui o trabalho, destacando as principais contribuições, limitações, sugestões para trabalhos futuros e o impacto da pesquisa no contexto das Infraestruturas de Dados Espaciais.