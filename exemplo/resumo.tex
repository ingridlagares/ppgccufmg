A utilização adequada de serviços de interoperabilidade em Infraestruturas de Dados Espaciais é fundamental para garantir a integração e compartilhamento dos dados de forma eficiente. A crescente necessidade de compartilhar e integrar dados espaciais levaram ao desenvolvimento de padrões específicos para APIs e Web Services no contexto geoespacial. Essa dissertação apresenta uma avaliação qualitativa e quantitativa entre a forma moderna e a legada de servir dados espaciais da Web. 
A metodologia para a avaliação qualitativa inclui o uso de um questionário baseado em estudos da literatura que usam métricas de avaliação de software para classificar, qualitativamente e do ponto de vista do desenvolvedor, as vantagens e desvantagens de cada abordagem.
Para a avaliação quantitativa foi realizada uma análise de desempenho das implementações compatíveis com a OGC API e OGC Web Service em um estudo de caso. Os resultados ajudam a orientar a seleção e implementação da abordagem mais adequada para a integração e compartilhamento de dados espaciais em Infraestruturas de Dados Espaciais, bem como desenvolver uma visão sobre o potencial impacto das normas técnicas, ora em evolução, para incorporação de APIs aos variados tipos de aplicações envolvendo dados geográficos.

\keywords{Geoinformática, Infraestrutura de Dados Espaciais, API, Web Services, OGC}
