Pelo fato  da OGC API ainda estar em desenvolvimento, com vários documentos ainda em fase de esboço, as referências abordadas para esse trabalho consideram três aspectos: 

\begin{itemize}
    \item Os documentos da OGC sobre a família de APIs e as discussões realizadas ao longo de \textit{Code Sprints}, organizadas para que os desenvolvedores participem da elaboração dos documentos.
    \item Os estudos existentes sobre as limitações percebidas do uso de Web Services no padrão OWS.
    \item Comparações sistemáticas dos diferentes tipos de abordagens para arquitetura e aplicação de APIs, com foco nas diferenças entre SOAP, o protocolo usado principalmente para Web Services, e REST, utilizado em outras abordagens de API incluindo a OGC API.
    \item Estudos de caso com implementações das padronizações mais recentes da OGC API.
\end{itemize}

\section{OGC Web Services}

% Para a dissertação: caberia aqui uma breve descrição do funcionamento do modelo de Web Services, para depois destacar as limitações

Vários pesquisadores de Sistemas de Informação Geográficos estudam as limitações atuais de IDEs e do modelo de serviços Web da OGC, que inclui WMS (Web Map Service), WFS (Web Feature Service), WCS (Web Coverage Service) e outros. Trabalhos como INSPIRE \citep{silvadiretiva} e TerraBrasilis \citep{fg2019terrabrasilis} estudam melhorias para avançar no uso de IDEs para analisar dados científicos. O TerraBrasilis está em conformidade com os padrões tradicionais da OGC (WFS e WMS), mas foi necessário desenvolver uma API própria para viabilizar estudos científicos com uso dos dados. % para a dissertação: mais detalhes aqui; que funções foram criadas para essa API, etc.

A estrutura da maioria das aplicações web geoespaciais são compatíveis e em conformidade com o serviço do antigo padrão da OGC, como, por exemplo: o INDE (Infraestrutura Nacional de Dados Espaciais) \citep{indewms}, a IDE da Plataforma Brumadinho UFMG \citep{brumadinho}, o Geoportal da IDE-BA \citep{ideba}, 
o IDE-BHGEO \citep{bhgeo}, o TerraBrasilis \citep{fg2019terrabrasilis} e o LindaGeo \citep{lindageo}. A existência de tantas IDEs nacionais relevantes destacam a importância deste trabalho.

\section{OGC API}

% para a dissertação: como na seção anterior, cabe aquyi uma breve descrição da arquitetura de API que a OGC propõe. 

De acordo com \cite{simoes2022datos}, a OGC API promove uma forma mais efetiva e popular de permitir o desenvolvimento de software de forma ágil, além de melhorar a encontrabilidade de dados espaciais. De acordo com a autora, o desenvolvimento de APIs REST é facilitado por ser flexível, autodocumentado, multiparte e por aproveitar de práticas web atuais. \cite{aihkisalo2012latencies} também concluem que uma API REST é mais fácil de desenvolver e implementar do que Web Services, uma vez que o protocolo REST está intimamente relacionado ao protocolo HTTP, tornando-se mais amigável ao ambiente Web.

Em um estudo de caso utilizando OGC API em uma aplicação, \cite{simoes2022serving} afirma que a OGC API ainda preciso de suporte de OGC Web Service, pois não está madura o suficiente e ainda não existem implementações o suficiente. O antigo padrão para servir metadados, por exemplo, ainda é mais seguro que o novo padrão, principalmente no que diz respeito à compatibilidade com grandes organizações responsáveis pela catalogação de metadados de instituições, países e continentes inteiros. Porém, considerando o aspecto de desenvolvimento, a autora afirma que com a nova família de padrões as APIs irão permitir uma melhor integração com outras tecnologias, irá facilitar a encontrabilidade de dados através de máquinas de busca como o Google, e o uso de OpenAPI poderá permitir documentação de APIs portáveis e auto-contidas.

% para a dissertação: algum trab rel que fale de limitações e do estado de desenvolvimento? Ou ref a algo online que indique o que está ainda evoluindo e como

%Em relação a documentação sobre bons princípios e requisitos para publicação de dados espaciais na Web, existem diversos autores que tratam essas questões com o objetivo de conduzir a elaboração de futuras padronizações  e produtores de geoinformação \cite{van2019best, dataweb,masser2019geographic,inspire}.

\section{Análises comparativas entre Web Services e API}
  
Foram consultados trabalhos de pesquisa que comparam o uso de diferentes tipos de abordagens para arquitetura e aplicação de APIs, com foco nas diferenças entres SOAP, o protocolo usado principalmente para Web Services, e REST, utilizado em outras abordagens de API incluindo a OGC API.  % para a dissertação: avaliar a viabilidade e o interesse de comparar a arquitetura de API da OGC com as de dados convencionais; talvez na seção anterior, ou até em uma seção própria.

\cite{mumbaikar2013web} desenvolveu duas aplicações com as mesmas funcionalidades, uma servindo protocolo REST e outra SOAP, para analisar a diferença de performance entre elas. Os resultados obtidos pelo autor foram em termos de tamanho de mensagem e tempo requerido para o processamento em dois tipos diferentes de requisições - com float e string, e com multimídia. Já Kishor Wagh \cite{wagh2012comparative} realiza uma comparação com métricas não-mensuráveis entre aplicações mobile SOAP e REST, visando as características arquiteturais de ambos os protocolos.

\citep{tihomirovs2016comparison} reuniram uma série de estudos relacionados aos protocolos REST e SOAP. O artigo sumariza vários trabalhos que avaliam métricas para comparar os protocolos. O autor utilizou uma ferramenta para encontrar todas as fontes que possuem como palavra-chave SOAP, REST e seus sinônimos. A partir disso foram aplicados critérios de exclusão dos artigos com base na sua relevância, idioma e abordagem. Foram analisados quatorze artigos, que trouxeram resultados relevantes de acordo com cada categoria de métrica - custo, esforço para desenvolver, linhas de código, velocidade de execução, memória, erros, funcionalidade, qualidade, complexidade, eficiência, confiabilidade e manutenibilidade.  

Em 2011, \citep{lopez2011review} executaram uma revisão nos Web Services em conformidade com a OGC na Europa.  O trabalho visava rastrear endpoints públicos de serviços geoespaciais e encontrou diversos desafios. O resultado comparou a quantidade de Web Service públicos encontráveis e não encontráveis através das máquinas de busca, sendo somente 55,4\% de serviços públicos encontráveis por máquina de busca. O trabalho visou apenas serviços compatíveis com a OGC Web Service e não comparar métricas de performance.

Os trabalhos de pesquisa encontrados que comparam os protocolos SOAP e REST não consideram serviços de geoinformação, portanto esta dissertação visa expandir esses resultados para Sistemas de Informação Geográfica. %% elaborar mais este argumento -- dizer por que ou em que os dados geo modificariam as comparações acima. 